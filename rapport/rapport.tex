\documentclass[11pt, a4paper]{article}

% --- PAQUETS ESSENTIELS ---
\usepackage[utf8]{inputenc} % Encodage
\usepackage[T1]{fontenc}    % Encodage de la police
\usepackage[french]{babel}  % Support de la langue française
\usepackage{hyperref}       % Liens cliquables
\usepackage{graphicx}       % Pour insérer les images (captures d'écran)
\usepackage{enumitem}       % Pour personnaliser les listes
\usepackage{parskip}        % Pour sauter des lignes entre les paragraphes au lieu de les indenter

% --- GESTION DES MARGES (Ajustez si nécessaire pour tenir sur 2 pages) ---
\usepackage[margin=2.5cm]{geometry} % Marge standard : 2.5cm

% --- TITRE ET INFORMATIONS ---
\title{\textbf{Rapport de Projet : FEUR (Bases du Web)}}
\author{Votre Nom et Prénom \\ ENSIIE FISA FC 2025-2026}
\date{Novembre 2025} % Mettez la date de rendu

\begin{document}

\maketitle
\thispagestyle{empty} % Supprime le numéro de page sur la page de titre (si vous en faites une)

% -------------------------------------------------------------------------
% SECTION 2 : TECHNOLOGIES ET ARCHITECTURE (CONTENU ÉVALUÉ)
% -------------------------------------------------------------------------
\section*{1. Technologies Utilisées et Architecture}

\subsection*{1.1. Architecture Technique}
Le projet FEUR, inspiré de Twitter, est implémenté avec une architecture client-serveur moderne :
\begin{itemize}
    \item \textbf{Client (Frontend)} : \textbf{Flutter} (Dart), choisi pour sa flexibilité multiplateforme (Web/Mobile).
    \item \textbf{Serveur (Backend)} : \textbf{Rust Axum}, utilisé pour sa performance et sa gestion des traitements asynchrones.
    \item \textbf{Base de Données} : \textbf{PostgreSQL}, une base de données relationnelle robuste.
\end{itemize}

\subsection*{1.2. Conformité aux Contraintes et Justification du Dépassement}
Les contraintes du sujet ont été respectées sur la partie serveur (10 fichiers) et la base de données (3 tables).
\begin{itemize}
    \item \textbf{Base de données} : \textbf{3 tables} SQL (\texttt{users}, \texttt{posts}, \texttt{user\_likes}).
    \item \textbf{Fichiers Client} : La limite de 15 fichiers non binaires est dépassée (environ 19 fichiers \texttt{.dart} et 2 fichiers \texttt{.arb}). Ce dépassement est justifié par le choix de la technologie \textbf{Flutter} qui requiert une organisation stricte en couches (\texttt{api}, \texttt{models}, \texttt{widgets}, \texttt{l10n}) et l'inclusion des fichiers de localisation (\texttt{l10n/\*.arb}) pour la fonctionnalité obligatoire de traduction.
\end{itemize}

% -------------------------------------------------------------------------
% SECTION 3 : PRÉSENTATION DU SITE ET MANUEL D'UTILISATION
% -------------------------------------------------------------------------
\section*{2. Présentation du Site et Manuel d'Utilisation}

\subsection*{2.1. Fonctionnalités de FEUR}
FEUR permet aux utilisateurs authentifiés de partager des pensées courtes. Le site propose deux profils utilisateurs :
\begin{itemize}
    \item \textbf{Standard} : Création de compte, publication, parcours du fil, et fonctionnalité de "like" (augmentant le \texttt{likes\_count} du post).
    \item \textbf{Administrateur} : Capacités supplémentaires pour gérer les utilisateurs (modification/suppression d'attributs via la colonne \texttt{is\_admin}) et la suppression des publications non conformes.
\end{itemize}

\subsection*{2.2. Modèle de Données et Requêtes}
La page du fil d'actualité affiche un contenu généré par la base de données. L'accès à la base de données utilise deux types de requêtes :
\begin{itemize}
    \item \textbf{Lecture (\texttt{SELECT})} : Récupération du fil d'actualité (jointure \texttt{posts} et \texttt{users}).
    \item \textbf{Modification (\texttt{INSERT/DELETE})} : Une requête \texttt{INSERT} est utilisée dans la table \texttt{user\_likes} pour enregistrer un nouveau "like". Une requête \texttt{DELETE} est employée par les administrateurs pour retirer des publications ou des utilisateurs.
\end{itemize}


% -------------------------------------------------------------------------
% SECTION 4 : DÉTAILS TECHNIQUES ET CHOIX DE CONCEPTION
% -------------------------------------------------------------------------
\section*{3. Détails Techniques et Choix de Conception}

\subsection*{3.1. Implémentation du Mode Jour/Nuit et de la Traduction}
\begin{itemize}
    \item \textbf{Mode Jour/Nuit} : Le changement est \textbf{dynamique} (sans rechargement de page), géré par le système de thèmes de Flutter, le thème sélectionné étant stocké dans un service (ex. : un Provider) et appliqué instantanément à l'intégralité de l'interface.
    \item \textbf{Traduction} : La traduction Français/Anglais est gérée via les fichiers de localisation \texttt{app\_\*.arb} et le système d'internationalisation de Flutter, permettant le changement de langue par le bouton dédié.
\end{itemize}

\subsection*{3.2. Choix Technique : Sécurité et Gestion des Rôles}
Nous avons implémenté un \textbf{middleware d'authentification} en Rust Axum (\texttt{auth/middleware.rs}). Ce middleware est responsable de l'interception des requêtes vers les routes protégées. Il vérifie la validité du jeton \textbf{JWT} et, crucialement, l'attribut \texttt{is\_admin} pour garantir que seuls les administrateurs peuvent accéder aux \texttt{user\_handlers.rs} de modification et suppression des comptes/posts.

\clearpage % Force le passage à la page suivante

% -------------------------------------------------------------------------
% SECTION HORS COMPTE DE PAGES (LES CAPTURES)
% -------------------------------------------------------------------------
\section*{Annexes : Captures d'écran (Illustrations du Site)}

\subsection*{4.1. Page du Fil d'Actualité}
% Les deux images ci-dessous ne compteront pas dans le décompte des 2 pages
\subsubsection*{Mode Français / Jour}
\includegraphics[width=\textwidth]{img/accueil_fr_jour.png}
\subsubsection*{Mode Anglais / Nuit}
\includegraphics[width=\textwidth]{img/accueil_en_nuit.png}

% Répétez pour toutes les pages et tous les modes requis.

\end{document}
